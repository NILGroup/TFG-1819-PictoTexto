% !TeX encoding = ISO-8859-1


\chapter{Herramientas}
\label{cap:Herramientas}

\begin{resumen}
	A lo largo de este cap�tulo analizaremos las diferentes herramientas utilizadas para el desarrollo de este TFG.
\end{resumen}

%-------------------------------------------------------------
\section{SimpleNLG-ES}
%-------------------------------------------------------------
\label{cap3:sec:SimpleNLG-ES}

Informaci�n sobre la biblioteca SimpleNLG-ES.  



%-------------------------------------------------------------
\section{Django}
%-------------------------------------------------------------
\label{cap3:sec:Django}

\subsection{�Qu� es Django?}
Es un framework de desarrollo web \footnote{conjunto de componentes que ayudan a desarrollar sitios web m�s r�pido y c�modo} de c�digo abierto escrito en python en su totalidad, adem�s respetando el patr�n de dise�o conocido como modelo-vista-controlador. La meta principal de esta aplicaci�n es facilitar la creaci�n de sitios web, para Django es muy importante el re-uso de c�digo, la conectividad y el desarrollo r�pido.\newline  
Django da soporte de base de datos permitiendo crear los modelos de datos necesarios y a trav�s de su API administrar y gestionar dicha base de datos. Adem�s concede al usuario la posibilidad de poder ejecutar sus propias consultas SQL.\newline
En la parte de servicios Web, Django incluye un servidor ligero que ofrece la posibilidad de realizar pruebas y trabajar en una etapa de desarrollo. Para una etapa m�s de producci�n ser�a m�s conveniente contar con otra aplicaci�n como puede ser Apache.

\subsection{�Por qu� elegimos Django?}
Debido a las caracter�sticas enumeradas en el apartado anterior y al elegir python como nuestro lenguaje de programaci�n toda la implementaci�n del trabajo se realizar� con Django.  \newline
El framework ofrece un servidor web en el que se podr� crear y gestionar una base de datos. En dicha base de datos estar�n contenidas las im�genes de cada pictograma y su correspondiente expresi�n, es decir permitir� asociar cada palabra con el pictograma que represente su mismo significado.\newline
Para llevar a cabo todas las consultas necesarias de la base de datos se implementar�n los tres m�todos de servicio web de tipo Restful: GET, POST, DELETE. El m�todo GET ser� el m�s utilizado ya que con �l se podr� obtener la mayor informaci�n sobre las im�genes y las palabras, dependiendo de la informaci�n que se requiera obtener se llamar�n a los diferentes m�todos.\newline
Todos estos resultados ser�n volcados en el servicio web creado a partir de Django, donde mediante un pictograma se conseguir� obtener su significado sint�ctico. 

  


