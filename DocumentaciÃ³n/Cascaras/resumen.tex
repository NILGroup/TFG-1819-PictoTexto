% !TeX encoding = ISO-8859-1
% +--------------------------------------------------------------------+
% | Copyright Page
% +--------------------------------------------------------------------+

\chapter*{Resumen}
La comunicaci�n es fundamental en la interacci�n social y posibilita la integraci�n y el intercambio. Sin embargo, hay personas con ciertas diversidades funcionales que afectan a su comunicaci�n verbal y deben usar sistemas aumentativos y alternativos de comunicaci�n (SAACS) como los pictogramas que les sirvan de apoyo en la comunicaci�n. No todo el mundo entiende la comunicaci�n mediante SAACS y esto impide a las personas que las necesitan para comunicarse poder tener una comunicaci�n efectiva, por m�s simple que pudiera llegar a ser lo que quieren decir.
Por todo ello, es necesario desarrollar recursos que permitan a las personas con alg�n tipo de discapacidad integrarse en el medio social y mejorar su calidad de vida.
En este TFG hemos desarrollado una aplicaci�n web para ayudar a aquellas personas con dificultades en la comunicaci�n, traduciendo textos escritos con pictogramas a texto escrito en lenguaje natural, m�s concretamente en castellano. En la actualidad, no existe ninguna aplicaci�n que haga una traducci�n de pictograma a texto, por lo que nuestro TFG puede ayudar a cubrir esa laguna.
Adem�s, la aplicaci�n ha sido implementada usando un dise�o basado en componentes, lo que permite a otros desarrolladores integrar cualquiera de nuestras componentes independientes en sus aplicaciones.
Por �ltimo, hemos realizado una evaluaci�n con el fin de comprobar la correcci�n de las traducciones realizadas por nuestra aplicaci�n y los resultados obtenidos son bastante alentadores aunque queda mucho margen de mejora. Podemos afirmar que con la aplicaci�n desarrollada en este TFG, se han establecido unas bases s�lidas sobre las que seguir trabajando para mejorar la cobertura y correcci�n de los textos traducidos. 
\clearpage  




 
\section*{Palabras clave}
   
\begin{itemize}
    \item Pictogramas.
    \item Aplicaci�n Web.
    \item Generaci�n Lenguaje Natural.
    \item Servicios web.
    \item Arasaac.
\end{itemize}

   


