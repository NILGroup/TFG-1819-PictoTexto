% !TeX encoding = ISO-8859-1
% +--------------------------------------------------------------------+
% | Copyright Page
% +--------------------------------------------------------------------+

\chapter*{Resumen}
Desde las primeras civilizaciones, el ser humano ha sentido la necesidad de comunicarse, en la actualidad esta necesidad es plasmada d�a a d�a en Internet: redes sociales, blogs... el hombre del siglo XXI, es m�s social y por ese  motivo la necesidad de comunicarse es mas importante que nunca. Sin embargo en nuestra sociedad hay personas con ciertas diversidades funcionales que les impiden poder tener una comunicaci�n efectiva, por m�s simple que pudiera llegar a ser lo que quieren decir. Por esta raz�n nacen los sistemas alternativos y aumentativos de comunicaci�n. 

Gracias a la aparici�n de diferentes alternativas de comunicaci�n, como pueden ser los pictogramas. Se ha ayudado a estas personas a lograr que puedan entender aquello que otra persona le est� queriendo transmitir. Aunque este tipo de comunicaci�n sigue siendo muy limitada e incluso a veces reduci�ndose dicha comunicaci�n a personas que poseen el mismo tipo de diversidad.\\
Por todo ello, es necesario desarrollar una herramienta que facilite a los usuarios de los pictogramas la comunicaci�n con las personas de su alrededor.\\

Este trabajo se centra en desarrollar una serie de servicios web que ayuden a aquellas personas con dificultades en la comunicaci�n, traduciendo los pictogramas que ellos utilizan a lenguaje natural y as� facilitar el intercambio de mensajes entre ellas y las personas de su alrededor. 
Este traductor esta implementado en una aplicaci�n web accesible para cualquier persona que lo desee usar, dicha aplicaci�n consta de un buscador de pictogramas y un tablero de pictogramas donde ir a�adiendo la secuencia de im�genes que queramos traducir. Por �ltimo, solo hace falta dar al bot�n de traducci�n para que se genere la frase en lenguaje natural.
   




 
\section*{Palabras clave}
   
\begin{itemize}
    \item Pictogramas.
    \item Generaci�n Lenguaje Natural.
    \item Servicios web.
    \item Arasaac.
    \item Metodolog�as �giles.
    \item Integraci�n Digital.
    \item Sistemas aumentativos y alternativos de comunicaci�n.
\end{itemize}

   


