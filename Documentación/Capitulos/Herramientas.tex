% !TeX encoding = ISO-8859-1


\chapter{Herramientas}
\label{cap:Herramientas}

A lo largo de este cap�tulo analizaremos las diferentes herramientas utilizadas para el desarrollo de este TFG.

%-------------------------------------------------------------
\section{Django}
%-------------------------------------------------------------
\label{cap3:sec:Django}

\subsection{�Qu� es Django?}
Es un framework de desarrollo web\footnote{https://tutorial.djangogirls.org/es/django/} de c�digo abierto escrito en Python, respetando el patr�n de dise�o conocido como Modelo-Vista-Controlador (M-V-C). La meta principal de esta aplicaci�n es facilitar la creaci�n de sitios web. En Django es muy importante el re-uso de c�digo, la conectividad y el desarrollo r�pido.\\  
\begin{enumerate}
\item Django da soporte de base de datos permitiendo crear los modelos de datos necesarios y a trav�s de su API administrar y gestionar dicha base de datos. Adem�s concede al usuario la posibilidad de poder ejecutar sus propias consultas SQL.
\item En la parte de servicios Web, Django incluye un servidor ligero que ofrece la posibilidad de realizar pruebas y trabajar en una etapa de desarrollo. Para una etapa m�s de producci�n ser�a m�s conveniente contar con otra aplicaci�n como puede ser Apache.
\end{enumerate}

\section{GSON}
%-------------------------------------------------------------
\label{cap3:sec:GSON}
GSON es una librer�a open-source la cu�l nos permite convertir nuestros objetos Java en JSON o viceversa, esta librer�a es muy importante ya que para lenguaje en Java no existen archivos JSON.
Al haber encapsulado el Servicio web en un formato REST, esto obliga a que nuestro Servicio tenga que recibir obligatoriamente un JSON.

En este momento es donde se utiliza la librer�a GSON la cu�l es un framework que contiene en su interior el m�todo "from JSON", este m�todo es vital para la construcci�n de nuestro Servicio web,�nicamente necesita recibir el dato mandado por el usuario y el tipo de clase al que hace referencia.

Una vez obtenidos los datos mencionados, la propia librer�a se encarga de realizar el resto de la conversi�n para que el Servicio web funcione correctamente.

\section{Apache Tomcat}
%-------------------------------------------------------------
\label{cap3:sec:Apache Tomcat} 
En el lado del servidor se ha decidido utilizar Apache Tomcat, donde se encuentra nuestro servidor web, lo que nos permite realizar la generaci�n de lenguaje natural.
Una vez el servicio recibe la frase introducida por el usuario, el servidor se encarga de analizar cada palabra de la frase y buscar los pictogramas correspondientes utilizando la API de Arasaac. El resultado obtenido es devuelto por el servidor mediante un GSON que relaciona cada palabra con su picto.\\ 
La versi�n de Apache utilizada en el servidor es la 9.0.14,para acceder al servicio de traducci�n del proyecto hay que introducir la siguiente direcci�n en el buscador: http://127.0.0.1:8000/127.0.0.1.




 


