% !TeX encoding = ISO-8859-1


\chapter{Herramientas}
\label{cap:Herramientas}

A lo largo de este cap�tulo analizaremos las diferentes herramientas utilizadas para el desarrollo de este TFG.

%-------------------------------------------------------------
\section{Django}
%-------------------------------------------------------------
\label{cap3:sec:Django}

\subsection{�Qu� es Django?}
Es un framework de desarrollo web\footnote{https://tutorial.djangogirls.org/es/django/} de c�digo abierto escrito en Python, respetando el patr�n de dise�o conocido como Modelo-Vista-Controlador (M-V-C). La meta principal de esta aplicaci�n es facilitar la creaci�n de sitios web. En Django es muy importante el re-uso de c�digo, la conectividad y el desarrollo r�pido.\\  
\begin{enumerate}
\item Django da soporte de base de datos permitiendo crear los modelos de datos necesarios y a trav�s de su API administrar y gestionar dicha base de datos. Adem�s concede al usuario la posibilidad de poder ejecutar sus propias consultas SQL.
\item En la parte de servicios Web, Django incluye un servidor ligero que ofrece la posibilidad de realizar pruebas y trabajar en una etapa de desarrollo. Para una etapa m�s de producci�n ser�a m�s conveniente contar con otra aplicaci�n como puede ser Apache.
\end{enumerate}


  


