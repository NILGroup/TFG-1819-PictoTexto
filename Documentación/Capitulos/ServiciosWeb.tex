% !TeX encoding = ISO-8859-1

\chapter{Servicios Web}
\label{cap:serviciosWeb}


	
A lo largo de este capitulo trataremos de explicar los diferentes servicios web, que hemos implementado, hablando sobre tecnolog�as y funciones que desarrollan dentro de nuestro sistema.
%------------------------------------------------------------------
\section{Servicio de Generaci�n de Lenguaje Natural}
%-------------------------------------------------------------------
\label{cap4:sec:ServicioGLN}
Como ya hemos hablado anteriormente, hemos utilizado el framework SimpleNLG-ES\footnote{https://github.com/citiususc/SimpleNLG-ES}, para la fase de generaci�n del lenguaje. Este framework est� escrito en Java, lo que impide su utilizaci�n dentro de nuestro servicio web central desarrollado en Python 3\footnote{https://docs.python.org/3/}.

Este Servicio Web utiliza Apache TomCat\footnote{http://tomcat.apache.org/} como contenedor, para encapsular el framework SimpleNLG-ES, y poder ser accesible desde el servicio web principal o cualquier otro servicio web del proyecto IDiLyCo\footnote{http://nil.fdi.ucm.es/index.php?q=projects/idilyco}.  

Este servicio web recibe un JSON con los datos necesarios para formar una frase, el cual gracias a GSON visto en \ref{cap3:sec:GSON} podemos obtener una clase Java con la cual formaremos la frase en lenguaje natural, la cual devolveremos en la respuesta del servicio.

%------------------------------------------------------------------
\section{Micro Servicios de Pictogramas}
%-------------------------------------------------------------------
\label{cap4:sec:MicroServicios}

Para la obtenci�n de los pictogramas as� como su significado se han implementado una serie de micro servicios que hacen uso de la API de ARASAAC\footnote{https://beta.arasaac.org/developers/api}. Estos servicios se han desarrollado usando el FrameWork DJango, del cual ya hemos hablado anteriormente.
A continuaci�n pasaremos a desarrollar los microservicios desarrollados:
\begin{itemize}

\item Servicio de b�squeda: este servicio web recibe una palabra y devuelve un objeto en formato JSON, que contiene una lista de objetos cada uno con el identificador de un pictograma de la base de datos de ARASAAC y la URL al archivo de imagen alojado en los servidores de ARASAAC. 

\item Servicio de traducci�n: el segundo servicio web recibe el identificador de un pictograma y obtiene todos los significados de dicho pictograma utilizando para ello la API de ARASAAC.

\item Servicio de landing: este servicio sirve para cargar la aplicaci�n web, recibe la petici�n y carga la aplicaci�n front-end, desde la que el usuario podr� acceder a toda la funcionalidad. 

\end{itemize} 

