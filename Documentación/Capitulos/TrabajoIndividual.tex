% !TeX encoding = ISO-8859-1

\chapter{Trabajo Individual}
\label{cap6:trabajoInd}
A lo largo de este capitulo hablaremos del trabajo realizado por cada uno de los miembros del equipo, siguiendo la metodolog�a de trabajo explicado anteriormente en \ref{cap3:sec:Metodolog�adetrabajo}

%------------------------------------------------------------------
\section{Trabajo Individual}
%-------------------------------------------------------------------
\subsection{Jose Maria}
\label{cap6:sec:Jose Maria}

Lo primero que realice fue un estudio del estado del arte, para comenzar con la redacci�n de este documento. La investigaci�n inicial se centro en en el estudio de los SAACS, poniendo especial atenci�n en los pictogramas y los sistemas pictograficos. 

La b�squeda de informaci�n sobre los sistemas, se analizaron los sistemas pictograficos m�s extendidos y la comunicaci�n a trav�s de  pictogramas. La investigaci�n concluyo con el uso de los pictogramas de ARASAAC para el desarrollo de este Trabajo de Fin de Grado.

Uno de los requisitos indispensables de este proyecto, era la necesidad de incluir la aplicaci�n dentro de un Servicio Web, por lo cual investigue sobre cuales eran las diferentes opciones para crear un Servicio Web en Python, y dada su potencia, facilidad de uso y la gran comunidad decidimos utilizar Django. Una vez expuesto y estudiado su uso, desarrolle una peque�a demo t�cnica, para probar su utilidad y como primer acercamiento a la tecnolog�a. 

El siguiente paso necesario, era un estudio sobre los diferentes sistemas de generaci�n de lenguaje parte fundamental del proyecto. Investigue y documente la base te�rica de la generaci�n de lenguaje, una vez termino esto pase a la b�squeda de diferentes frameworks de generaci�n de lenguaje en espa�ol, tras la cual encontr� el framework SimpleNLg-ES, que es el  framework de generaci�n de lenguaje natural en espa�ol que encontr�. Una vez presentados los resultados desarrolle una demo utilizando dicho framework en un servicio web al cual se llama desde el servicio web de Django.

Una vez desarrollada la demo y con el aumento de los servicios web implementados, decid� refactorizar el c�digo y dise�e una arquitectura que fomentara la r�pida integraci�n de nuevos servicios, que adem�s sean completamente independientes entre ellos siguiendo el principio de responsabilidad �nica\footnote{https://devexperto.com/principio-responsabilidad-unica/} por el cual cada modulo debe de realizar una �nica cosa. 

Dado el limite del framework Django para la gesti�n de aplicaciones web desde el lado del cliente, desarrolle una aplicaci�n web fron-end, que despu�s integre con la aplicaci�n Django, para de esta manera obtener mayor funcionalidad y descargar parte del trabajo en parte del cliente.