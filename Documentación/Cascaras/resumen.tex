% !TeX encoding = ISO-8859-1
% +--------------------------------------------------------------------+
% | Copyright Page
% +--------------------------------------------------------------------+

\chapter*{Resumen}
Desde las primeras civilizaciones, el ser humano ha sentido la necesidad de comunicarse, en la actualidad esta necesidad es plasmada d�a a d�a en Internet: redes sociales, blogs... el hombre del siglo XXI, es m�s social y por ese  motivo la necesidad de comunicarse es mas importante que nunca. Sin embargo en nuestra sociedad hay personas con ciertas diversidades funcionales que les impiden poder tener una comunicaci�n efectiva, por m�s simple que pudiera llegar a ser lo que quieren decir. Por esta raz�n nacen los sistemas alternativos y aumentativos de comunicaci�n. 

Gracias a la aparici�n de diferentes alternativas de comunicaci�n, como pueden ser los pictogramas. Se ha ayudado a estas personas a lograr que puedan entender aquello que otra persona le est� queriendo transmitir. Aunque este tipo de comunicaci�n sigue siendo muy limitada e incluso a veces reduci�ndose dicha comunicaci�n a personas que poseen el mismo tipo de diversidad.\\
Por todo ello, es necesario desarrollar una herramienta que facilite a los usuarios de los pictogramas la comunicaci�n con las personas de su alrededor.\\

Este trabajo se centra en desarrollar una aplicaci�n web que ayude a aquellas personas con dificultades en la comunicaci�n, traduciendo los pictogramas que ellos utilizan a lenguaje natural y as� facilitar el intercambio de mensajes entre ellas y cualquier persona de su alrededor. Es importante mencionar que no existe ahora mismo ninguna aplicaci�n que ofrezca lo que se busca en este trabajo, la traducci�n de pictograma a texto.
Adem�s se han tratado de implementar una serie de servicios web que sean diferentes e independientes unos de otros y que puedan ser reutilizables por otros programadores.   
Este traductor esta pensado para ser una aplicaci�n web accesible para cualquier persona que lo desee usar, dicha aplicaci�n consta de un buscador de pictogramas y un tablero de pictogramas donde ir a�adiendo la secuencia de im�genes que queramos traducir, despu�s solo hace falta dar al bot�n de traducci�n para que se genere la frase en lenguaje natural.\\
Para finalizar, es importante mencionar que los resultados obtenidos de nuestro trabajo son bastante alentadores, ya que como he mencionado antes se ha implementado una aplicaci�n que hasta ahora no exist�a, por lo que a�n queda mucho margen de mejora en ella, pero si podemos afirmar que se han establecido unas bases s�lidas sobre las que seguir desarrollando la aplicaci�n. 
   




 
\section*{Palabras clave}
   
\begin{itemize}
    \item Pictogramas.
    \item Aplicaci�n Web.
    \item Generaci�n Lenguaje Natural.
    \item Servicios web.
    \item Arasaac.
    \item Metodolog�as �giles.
    \item Integraci�n Digital.
    \item Sistemas aumentativos y alternativos de comunicaci�n.
\end{itemize}

   


