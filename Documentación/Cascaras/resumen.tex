% !TeX encoding = ISO-8859-1
% +--------------------------------------------------------------------+
% | Copyright Page
% +--------------------------------------------------------------------+

\chapter*{Resumen}

El ser humano ha sentido la necesidad de comunicarse desde las primeras civilizaciones, en la actualidad esta necesidad es plasmada d�a a d�a en Internet, redes sociales, blogs... el hombre del siglo XXI, es m�s social que nunca y por ese  motivo la necesidad de comunicarse es mas importante que nunca. No todas las personas  tienen la misma capacidad de expresarse y o comunicarse, y por esta raz�n nacen los sistemas  alternativos de comunicaci�n. 

El uso de pictogramas es uno de estos sistemas, en los �ltimos a�os hemos visto una evoluci�n en la utilizaci�n de pictogramas en el �mbito educativo, en especial en ni�os con TEA (Trastorno del Espectro Autista) y trastornos del lenguaje en general como puedan ser Asperger, S�ndrome de Down, paralisis celebral, etc.

Por estos motivos hemos decidido crear un servicio web que permita al usuario traducir pictogramas a lenguaje natural.  

 
\section*{Palabras clave}
   
\noindent M�ximo 10 palabras clave separadas por comas

   


