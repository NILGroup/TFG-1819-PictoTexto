% !TeX encoding = ISO-8859-1
% +--------------------------------------------------------------------+
% | Copyright Page
% +--------------------------------------------------------------------+

\chapter*{Resumen}
La comunicaci�n es fundamental en la interacci�n social y posibilita la integraci�n y el intercambio. Sin embargo, hay personas con ciertas diversidades funcionales que afectan a su comunicaci�n verbal y deben usar sistemas SAACS que les sirvan de apoyo en la comunicaci�n. No todo el mundo entiende la comunicaci�n que emplea SAACS y esto les impide poder tener una comunicaci�n efectiva, por m�s simple que pudiera llegar a ser lo que quieren decir.

Por todo ello, es necesario desarrollar recursos que permitan a las personas con alg�n tipo de discapacidad insertarse en el medio social y mejorar su calidad de vida.
El objetivo de este TFG es desarrollar una aplicaci�n web que ayude a aquellas personas con dificultades en la comunicaci�n, traduciendo textos escritos con pictogramas a texto escrito en lenguaje natural, m�s concretamente en castellano. En la actualidad no existe ninguna aplicaci�n que haga una traducci�n de pictograma a texto.
Adem�s se ha implementado una aplicaci�n web sencilla, �til y basada en componentes diferentes e independientes unas de otras.\\
Los resultados obtenidos son bastante alentadores,aunque queda mucho margen de mejora. Pero si podemos afirmar que se han establecido unas bases s�lidas sobre las que seguir aumentando la cobertura y correcci�n de los textos traducidos. 
\clearpage  




 
\section*{Palabras clave}
   
\begin{itemize}
    \item Pictogramas.
    \item Aplicaci�n Web.
    \item Generaci�n Lenguaje Natural.
    \item Servicios web.
    \item Arasaac.
\end{itemize}

   


