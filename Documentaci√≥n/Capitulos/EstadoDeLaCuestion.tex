% !TeX encoding = ISO-8859-1

\chapter{Estado de la Cuesti�n}
\label{cap:estadoDeLaCuestion}

En el estado de la cuesti�n es donde aparecen gran parte de las referencias bibliogr�ficas del trabajo. Una de las formas m�s c�modas de gestionar la bibliograf�a en {\LaTeX} es utilizando \textbf{bibtex}. Las entradas bibliogr�ficas deben estar en un fichero con extensi�n \textit{.bib} (con esta plantilla se proporcionan 3, dos de los cuales est�n vac�os). Cada entrada bibliogr�fica tiene una clave que permite referenciarla desde cualquier parte del texto con los siguiente comandos:

\begin{itemize}
\item Referencia bibliografica con cite: \cite{ldesc2e}
\item Referencia bibliogr�fica con citep: \citep{notsoshort}
\item Referencia bibliogr�fica con citet: \citet{latexAPrimer}
\end{itemize}

Es posible citar m�s de una fuente, como por ejemplo \citep{latexCompanion,LaTeXLamport,texKnuth}

Despu�s, latex se ocupa de rellenar la secci�n de bibliograf�a con las entradas \textbf{que hayan sido citadas} (es decir, no con todas las entradas que hay en el .bib, sino s�lo con aquellas que se hayan citado en alguna parte del texto).

Bibtex es un programa separado de latex, pdflatex o cualquier otra cosa que se use para compilar los .tex, de manera que para que se rellene correctamente la secci�n de bibliograf�a es necesario compilar primero el trabajo (a veces es necesario compilarlo dos veces), compilar despu�s con bibtex, y volver a compilar otra vez el trabajo (de nuevo, puede ser necesario compilarlo dos veces). 


\begin{resumen}
	A lo largo de este capitulo trataremos los aspectos m�s important

\end{resumen}



%------------------------------------------------------------------
\section{SAAC}
%-------------------------------------------------------------------
\label{cap2:sec:SAAC}


Los Sistemas Aumentativos y Alternativos de comunicaci�n son distintas formas de expresar el lenguaje hablado, que tienen como objetivo aumentar y/o compensar las dificultades de muchas personas con dificultades para conseguir una comunicaci�n verbal funcional. 
La comunicaci�n es una necesidad, y en casos de autismo o par�lisis cerebral donde el lenguaje oral est� gravemente limitado, la utilizaci�n de sistemas de comunicaci�n de lenguaje no verbales, como los pictogramas, sustituyen o sirven de apoyo al lenguaje verbal. 
Lo que se pretende con los SAAC es conseguir una comunicaci�n funcional, adecuada y generalizable, que permita al individuo expresarse y alcanzar una mayor integraci�n en su entorno social. Los SAAC son una herramienta de calidad de vida al otorgar al individuo de una v�a para comunicarse.
Cabe destacar que los SAAC no son de uso exclusivo de las personas con des�rdenes en la comunicaci�n, sino que son utilizados por todos a diario por ejemplo gestos, se�ales de trafico, etc.

Existen dos tipos de SAAC.
\begin{enumerate}
\item Sin ayuda, aquellos sistemas que carecen de soporte f�sico, ejemplo el lenguaje de signos.
\item Con ayuda, aquellos sistemas que necesitan de un soporte f�sico, ejemplo los pictogramas.

\end{enumerate}
 
A principio de los a�os 80 se empiezan a dar mayor rigor a los SAAC y aparecen los primeros programas inform�ticos, con predicci�n de palabras o de pictogramas.
 
%------------------------------------------------------------------
\section{Pictogramas}
%-------------------------------------------------------------------
\label{cap2:sec:Pictogramas}

Desarrollado en 1981 por Mayer y Johnson se caracteriza por una facilidad de interpretaci�n.
Sistema de lenguaje con ayuda o asistida requiere un apoyo f�sico 
INFORMACION SOBRE LOS PICTOGRAMAS
