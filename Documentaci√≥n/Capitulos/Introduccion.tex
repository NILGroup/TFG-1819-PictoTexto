% !TeX encoding = ISO-8859-1

\chapter{Introducci�n}
\label{cap:introduccion}

\chapterquote{Frase c�lebre dicha por alguien inteligente}{Autor}

\section{Motivaci�n}

La comunicaci�n forma parte fundamental para el correcto desarrollo social y humano, en los �ltimos a�os los Sistemas Aumentativos y Alternativos de Comunicaci�n SAACs han cobrado especial fuerza para facilitar la comunicaci�n de usuarios con diversidad funcional.
Aunque la aparici�n de diferentes recursos software, como traductores de lenguaje natural a pictogramas, ha ayudado a estas personas a poder comunicarse en su d�a a d�a, la realidad es que la comunicaci�n sigue estando muy limitada y en ocasiones se reduce a personas con el mismo tipo de diversidad funcional. Este es el caso de usuarios cuya v�a de comunicaci�n son los pictogramas, ya que el resto de usuarios desconoce su significado lo que produce que la comunicaci�n queda notablemente reducida o en muchos casos esta sera completamente imposible.
Esta es la raz�n, por la cual existe la necesidad de crear una herramienta que ayuden a la comunicaci�n de los usuarios de pictogramas con su entorno, una software que sea capaz de traducir pictogramas a lenguaje natural, esta herramienta software proporcionara una via de comunicaci�n derribando barreras y ayudando a la inclusi�n social de personas que por diversidad funcional est�n en riesgo de exclusi�n.

El objetivo de este TFG es el de crear un traductor de pictogramas a texto, que permita a usuarios cuya principal via de comunicaci�n son los pictogramas comunicarse de manera fluida.Para ello se implementar�n servicios web que permitan la interpretaci�n
(no de traducci�n literal) de pictogramas a texto. Los resultados de este trabajo ayudar�n
a las personas que presentan problemas de inclusi�n debido a la necesidad de usar
pictogramas para comunicarse.

\subsection{Objetivos}
Si quieres cambiar el \textbf{estilo del t�tulo} de los cap�tulos, abre el fichero \verb|TeXiS\TeXiS_pream.tex| y comenta la l�nea \verb|\usepackage[Lenny]{fncychap}| para dejar el estilo b�sico de \LaTeX.

Si no te gusta que no haya \textbf{espacios entre p�rrafos} y quieres dejar un peque�o espacio en blanco, no metas saltos de l�nea (\verb|\\|) al final de los p�rrafos. En su lugar, busca el comando  \verb|\setlength{\parskip}{0.2ex}| en \verb|TeXiS\TeXiS_pream.tex| y aumenta el valor de $0.2ex$ a, por ejemplo, $1ex$.

El siguiente texto se genera con el comando \verb|\lipsum[2-20]| que viene a continuaci�n en el fichero .tex. El �nico prop�sito es mostrar el aspecto de las p�ginas usando esta plantilla. Quita este comando y, si quieres, comenta o elimina el paquete \textit{lipsum} al final de \verb|TeXiS\TeXiS_pream.tex|

\subsubsection{Texto de prueba}
