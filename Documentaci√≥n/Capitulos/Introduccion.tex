% !TeX encoding = ISO-8859-1

\chapter{Introducci�n}
\label{cap:introduccion}

\chapterquote{Frase c�lebre dicha por alguien inteligente}{Autor}

\section{Motivaci�n}

La comunicaci�n es una parte fundamental del correcto desarrollo social y humano, pero no es igual de accesible para todos, para personas con diferentes diversidades funcionales existen muchos obst�culos para una comunicaci�n correcta.\\
La aparici�n de diferentes v�as de comunicaci�n, como por ejemplo los pictogramas, ha supuesto un gran avance para lograr una comunicaci�n efectiva, apoyados gracias a  la aparici�n de diferentes recursos software, como traductores de lenguaje natural a pictogramas, han ayudado a estas personas a poder comunicarse en su d�a a d�a, pero la realidad es que la comunicaci�n sigue estando muy limitada y en ocasiones se reduce a personas con el mismo tipo de diversidad funcional. Este es el caso de usuarios cuya v�a de comunicaci�n son los pictogramas, ya que el resto de usuarios desconoce su significado lo que produce que la comunicaci�n quede notablemente reducida o en muchos casos sea completamente imposible.
Esta es la raz�n, por la cual existe la necesidad de crear una herramienta que ayude a la comunicaci�n de los usuarios de pictogramas con su entorno, una software que sea capaz de traducir pictogramas a lenguaje natural, esta herramienta software proporcionara una v�a de comunicaci�n derribando barreras y ayudando a la inclusi�n social de personas que por diversidad funcional est�n en riesgo de exclusi�n.

El objetivo de este TFG es el de crear un traductor de pictogramas a texto, que permita a usuarios cuya principal v�a de comunicaci�n son los pictogramas comunicarse de manera fluida, para ello crearemos un servicio web que permita la interpretaci�n
(no la traducci�n literal)de pictogramas a texto. Los resultados de este trabajo ayudar�n a las personas que presentan problemas de inclusi�n debido a la necesidad de usar
pictogramas para comunicarse.

\subsection{Objetivos}

El objetivos principal de este trabajo, es la creaci�n de un traductor de pictogramas a lenguaje natural, buscando la interpretaci�n del mensaje y no una traducci�n literal. Un producto que sea accesible para todos, buscando la facilidad de uso a trav�s de una interfaz c�moda accesible e intuitiva, para de esta manera llegar al m�ximo publico posible.  \\

Buscamos tambi�n afianzar los conocimientos adquiridos durante el grado, y aplicar esos mismos conocimientos utilizando un dise�o y visi�n, sobre el proceso y el producto orientada a la mantenibilidad del mismo, una gran experiencia de uso  y durante la realizaci�n del producto adquirir nuevos conocimientos, gracias al desarrollo de un trabajo con impacto real. Para ello desarrollaremos un servicio web, que permita introducir un conjunto de pictogramas y devuelva al usuario, una interpretaci�n en lenguaje natural del significado del mensaje introducido. Integraremos este servicio dentro de una aplicaci�n web accesible para todo el mundo.


\subsection{Estructura del trabajo}


