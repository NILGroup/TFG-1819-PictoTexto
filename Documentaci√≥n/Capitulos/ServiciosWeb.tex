% !TeX encoding = ISO-8859-1

\chapter{Servicios Web}
\label{cap:serviciosWeb}


	
A lo largo de este capitulo trataremos de explicar los diferentes servicios web, que hemos implementado, hablando sobre tecnolog�as y funciones que desarrollan dentro de nuestro sistema.
%------------------------------------------------------------------
\section{Servicio de Generaci�n de Lenguaje Natural}
%-------------------------------------------------------------------
\label{cap4:sec:ServicioGLN}
Como ya hemos hablado anteriormente, hemos utilizado el framework SimpleNLG-ES\footnote{https://github.com/citiususc/SimpleNLG-ES}, para la fase de generaci�n del lenguaje. Este framework est� escrito en Java, lo que impide su utilizaci�n dentro de nuestro servicio web central desarrollado en Python 3\footnote{https://docs.python.org/3/}.

Este Servicio Web utiliza Apache TomCat\footnote{http://tomcat.apache.org/} como contenedor, para encapsular el framework SimpleNLG-ES, y poder ser accesible desde el servicio web principal o cualquier otro servicio web del proyecto IDiLyCo\footnote{http://nil.fdi.ucm.es/index.php?q=projects/idilyco}.  

Este servicio web recibe un JSON con los datos necesarios para formar una frase, el cual gracias a GSON visto en \ref{cap3:sec:GSON} podemos obtener una clase Java con la cual formaremos la frase en lenguaje natural, la cual devolveremos en la respuesta del servicio.