% !TeX encoding = ISO-8859-1

\chapter{Servicios Web}
\label{cap:serviciosWeb}


	
A lo largo de este capitulo trataremos de explicar los diferentes servicios web, que hemos implementado, para desarrollar cada una de las funcionalidades de nuestra aplicaci�n.
%------------------------------------------------------------------
\section{Servicio de Generaci�n de Lenguaje Natural}
%-------------------------------------------------------------------
\label{cap4:sec:ServicioGLN}
Como ya hemos hablado anteriormente, hemos utilizado el framework SimpleNLG-ES\footnote{https://github.com/citiususc/SimpleNLG-ES}, para la fase de generaci�n del lenguaje. Este framework est� escrito en Java, lo que impide su utilizaci�n dentro de nuestro servicio web central desarrollado en Python 3.

Este Servicio Web utiliza Apache TomCat como contenedor, para encapsular el framework SimpleNLG-ES, y poder ser accesible desde el servicio web principal o cualquier otro servicio web del proyecto IDiLyCo\footnote{http://nil.fdi.ucm.es/index.php?q=projects/idilyco}.  

Este servicio web recibe un JSON con un simple array de palabras, tal y como podemos ver en la Figura \ref{fig:nlgRequest}. Esta petici�n puede ser parseada gracias al framework GSON visto en el Capitulo \ref{cap3:sec:GSON} para obtener una clase Java con la cual formaremos la frase en lenguaje natural, la cual se devuelve en la respuesta del servicio, en texto plano tal y como podemos ver en la Figura \ref{fig:nlgResponse}.
\figura{Bitmap/EstadoDeLaCuestion/nlgRequest}{width=260px}{fig:nlgRequest}{Petici�n al servicio de creaci�n del lenguaje}
\figura{Bitmap/EstadoDeLaCuestion/nlgResponse}{width=220px}{fig:nlgResponse}{Respuesta del servicio de creaci�n del lenguaje}

\section{Fases del proceso}

En est� secci�n vamos a explicar, las diferentes fases que transforman la entrada del servicio de generaci�n de lenguaje hasta obtener una frase valida.

\subsection{Preparaci�n}
Una vez el servidor recibe la petici�n Post, el frameword GSON se encarga de parsear la entrada para devolver un array de palabras, el cual se enviara al sistema generador de lenguaje.
Una vez el sistema recibe el array de palabras, lo primero que haces un procesamiento y clasificaci�n de todas las palabras seg�n categor�a l�xica:
Diferenciando entre:
\begin{enumerate}
\item Adjetivos.
\item Adverbios.
\item Determinantes.
\item Pronombres.
\item Nombre.
\item S�mbolos.
\item Conjunciones.
\item Preposiciones.
\item Complementos.
\item Verbos modales.
\item Complementos
\end{enumerate}
Esta clasificaci�n es autom�tica gracias a la funci�n:
\begin{center}
 NLGFactory.createWord("palabra", LexicalCategory.ANY)
\end{center}

Con el resultado de esta clasificaci�n podemos determinar la posici�n del verbo, con lo cual podemos dividir el array en aquellas palabras que formaran el sujeto y aquellas que formaran el predicado. Cabe destacar que ARAASAC devuelve los verbos en infinitivo y est� funci�n de clasificaci�n detecta los verbos como nombres. Este problema tiene soluci�n gracias a una peque�a l�gica que detecta si los nombres al ser configurados como verbos tienen conjugaciones.

\subsection{Sujeto}

Una vez tenemos el subarray de palabras ya clasificadas que formaran el sujeto se pasa a procesar el sujeto de la frase. En este proceso se configura un determinante en caso de que sea necesario, por defecto el determinante es \textit{ el o la}. Adem�s se reconoce el genero y numero del nombre principal del sujeto, para proceder a la configuraci�n de los mismos de todo el sujeto. 

%------------------------------------------------------------------
\section{Micro Servicios de Pictogramas}
%-------------------------------------------------------------------
\label{cap4:sec:MicroServicios}

Para la obtenci�n de los pictogramas as� como su significado se han implementado una serie de micro servicios que hacen uso de la API de ARASAAC\footnote{https://beta.arasaac.org/developers/api}. Estos servicios se han desarrollado usando el FrameWork DJango, del cual ya hemos hablado anteriormente.
A continuaci�n pasaremos a desarrollar los microservicios desarrollados:
\begin{itemize}

\item Servicio de b�squeda: este servicio web, recibe una palabra a traves de una petici�n GET tal y como podemos ver en la Figura \ref{fig:pictoFinderRequest}, y devuelve un objeto en formato JSON. Este servicio primero lanza una petici�n GET a un servicio web externo de ARAASAC, el cual devuelve una lista de pictogramas. Esta respuesta es tratada, para crear un objecto que contendr� una lista de objetos cada uno con el identificador de un pictograma de la base de datos de ARASAAC, la URL al archivo de imagen alojado en los servidores de ARASAAC y el significado de ese pictograma. Una vez toda la informaci�n es tratada, se genera un JSON que se devuelve en la respuesta del servicio.
\figura{Bitmap/EstadoDeLaCuestion/pictoFinderRequest}{width=300px}{fig:pictoFinderRequest}{Petici�n al servicio de b�squeda de pictogramas}
\figura{Bitmap/EstadoDeLaCuestion/pictoFinderResponse}{width=220px}{fig:pictoFinderResponse}{Respuesta del servicio de b�squeda de pictogramas para la palabra casa}

\item Servicio de traducci�n: el segundo servicio web recibe el identificador de un pictograma y obtiene todos los significados de dicho pictograma utilizando para ello la API de ARASAAC.

\item Servicio de landing: este servicio sirve para cargar la aplicaci�n web, recibe la petici�n y carga la aplicaci�n front-end, desde la que el usuario podr� acceder a toda la funcionalidad. 

\end{itemize} 

